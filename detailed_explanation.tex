-------------------------------------------------------------------------------
ENVIRONMENT VARIABLES 🌍🔑
-------------------------------------------------------------------------------
Think of environment variables as secret notes your operating system keeps to configure how programs behave. They store important information 
like file paths, API keys, and system settings without hardcoding them into programs.

🔹 Examples:
PATH: Tells the OS where to find executables.
HOME: Points to your user’s home directory.
AWS_SECRET_KEY: Stores AWS credentials securely.

🔹 Accessing them:
Linux/macOS: echo $VAR_NAME
Windows: echo %VAR_NAME%
Python: import os; print(os.getenv("VAR_NAME"))
Why use them? Because they make software portable, configurable, and more secure! 🚀

-------------------------------------------------------------------------------
EXECUTABLE FILES🏃💾
-------------------------------------------------------------------------------
Imagine an executable file as a magic scroll 📜 that, when opened, makes the computer follow its commands instantly! These files contain 
machine-readable instructions that your system can run without needing an external interpreter.

🔹 Types of Executables:
Windows: .exe (chrome.exe, notepad.exe)
Linux/macOS: No extension (/bin/bash, /usr/bin/python3)
Scripts (require an interpreter): .py, .sh, .bat

🔹 How to run them?
Windows: Double-click or run via cmd (.\file.exe)
In short, executables bring software to life! 💡🔥

------------------------------------------------------------------------------
DOES EVERY APPLICATION HAS EXECUTABLES?🤔💻
------------------------------------------------------------------------------
Yes... but not always in the form you expect! While most software has an executable component, some rely on scripts, interpreters,
or virtual environments to function.

🔹 Different Scenarios:
✔️ Standalone Apps: Have direct executables (chrome.exe, firefox, bash)
✔️ Interpreted Languages: Python (script.py) needs python to run
✔️ Packages & Libraries: Might provide CLI commands (pip, gunicorn)
✔️ Compiled Programs: C, Go, Rust create .exe or binary files

So, while every runnable program depends on executables in some way, not all applications come as a single executable file. Sometimes,
they need helpers like interpreters or runtime environments to work their magic! 🪄✨

------------------------------------------------------------------------------
AUGMENTAION TECHNIQUES🎨🤖
------------------------------------------------------------------------------
Augmentation techniques are methods used to artificially expand and diversify a dataset by applying transformations to existing data. 
These techniques are widely used in computer vision, NLP, and audio processing to improve model generalization and prevent overfitting.

📸 Image Augmentation (For Computer Vision) (FOCUS OF THIS PROJECT)
1️⃣ Rotation – Rotating an image by a small degree.
2️⃣ Flipping – Mirroring the image horizontally or vertically.
3️⃣ Cropping & Scaling – Zooming in or out while keeping the subject in focus.
4️⃣ Brightness & Contrast Adjustment – Changing lighting conditions.
5️⃣ Adding Noise – Introducing random variations to make models more robust.
6️⃣ CutMix & MixUp – Merging parts of multiple images to create hybrid samples.

📝 Text Augmentation (For NLP)
1️⃣ Synonym Replacement – Replacing words with synonyms.
2️⃣ Back Translation – Translating text into another language and back.
3️⃣ Word Order Shuffling – Randomly reordering words.
4️⃣ Character-level Noise – Introducing typos and missing letters.
5️⃣ Sentence Paraphrasing – Using AI to rewrite sentences.

🔊 Audio Augmentation (For Speech & Sound)
1️⃣ Pitch Shifting – Modifying the pitch of the audio.
2️⃣ Time Stretching – Speeding up or slowing down speech.
3️⃣ Adding Background Noise – Simulating real-world noise conditions.
4️⃣ Volume Adjustment – Changing loudness.
5️⃣ Spectrogram Masking – Masking parts of the spectrogram(a visual representation of the frequency content of a signal over time.
    It shows how the frequency (pitch) of a sound or signal varies across time) to improve robustness.

🔥 Why Use Augmentation?
✅ Reduces overfitting.
✅ Makes models more robust to real-world variations.
✅ Allows training with less data.
